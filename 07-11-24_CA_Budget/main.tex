%----------------------------------------------------------------------------------------
%	PACKAGES AND THEMES
%----------------------------------------------------------------------------------------
\documentclass{beamer}
\usetheme[nomail, nogauge, delaunay, amurmapleblack]{Amurmaple}
\usepackage[french]{babel}
\usepackage{hyperref}
\usepackage{graphicx} % Allows including images
\usepackage{booktabs} % Allows the use of \toprule, \midrule and \bottomrule in tables
\usepackage{animate}
\usepackage{tikz}
\usepackage{graphicx}
%----------------------------------------------------------------------------------------
%	TITLE PAGE
%----------------------------------------------------------------------------------------

% The title
\title[Demande de budget]{Demande de budget au CA élèves}

\author[Centrale Mathématiques] {Raphaël Casanova}
\institute[NTU] % Your institution may be shorthand to save space
{
}
\date{\today} % Date, can be changed to a custom date


%----------------------------------------------------------------------------------------
%	PRESENTATION SLIDES
%----------------------------------------------------------------------------------------

\begin{document}

\begin{frame}
    % Print the title page as the first slide
    \titlepage
\end{frame}

\sepframe[title = {Sommaire}]

%------------------------------------------------
\section{Introduction rapide}
%------------------------------------------------
\begin{frame}
  \framesection{Projet de l'association :}
  Qui dit année de lancement dit que tout est à construire. Ainsi, l'objectif principal de ce premier mandat est d'installer l'association dans l'univers de Centrale. 
  Les demandes effectuées cette année constitue plus de "l'achat de ressource" qui pourront être utile aux plus grands nombres mais également au bureau de l'association pour réaliser ces supports, corrigés, notes explicatives. 
  
  \framesection{Chiffres de la demande :}
  \begin{itemize}
    \item 6 livres traitant d'Algèbre, de Topologie, de Mathématiques appliquées, d'Analyse
    \item Total de 213,76€
  \end{itemize}
\end{frame}

%------------------------------------------------
\section{Présentation des achats}
%------------------------------------------------
\begin{frame}
  \framesection{En analyse :}
  \begin{itemize}
    \item "Analyse Complexe", Hervé et Martine Quéffelec : une référence dans le domaine qui est étudiée par les L3 en convention avec Centrale mais également très utilisée en physique
    \item "An introduction to Laplace Transform and Fourier Series", Phil Dyke : référence également dans le domaine et un support de cours intéressant pour les transformées de Laplace et Fourier qui sont traitées à l'ITEEM et Centrale
  \end{itemize}
\end{frame}

\begin{frame}
  \framesection{En topologie :}
  \begin{itemize}
    \item "Topologie Algèbrique", Yves Félix et Daniel Tanré : matière présente au master de mathématiques de la convention Centrale et direct application des cours d'algèbres et de topologie pour les L3. Ce livre permettra de proposer des corrigés aux TDs de ces différentes matières pour aider les centraliens.
    \item "Topological Quantum", Steven H. Simon : Livre de mathématiques appliquées à la physique fondamentale, pourra s'avérer intéressant pour les étudiants souhaitant approfondir les cours de physique moderne, permettra également la rédaction de notes à ce sujet. 
  \end{itemize}
\end{frame}

\begin{frame}
\framesection{En algèbre :}
\begin{itemize}
  \item Polycopié de "Galois Theory", Ian Stewart : Référence dans le sujet, matière étudiée en M1 convention Centrale, une théorie fondamentale et riche des mathématiques
\end{itemize}
\end{frame}


%------------------------------------------------
\section{Conclusion rapide}
%------------------------------------------------
\begin{frame}
  \framesection{Conclusion rapide :}
  \begin{itemize}
  \item  Les prix indiqués sur les livres sont leur prix "neuf", dès lors, nous tenons à signifier que nous essaierons dans une démarche environnementale et économique de trouver ces livres d'occasions. 
  \item   L'achat de ces livres permettra d'étayer et de sourcer les animations mathématiques proposées par l'association, ces livres se verront être accessibles librement à tous les centraliens pour des emprunts. 
  \item   Ces livres ont été sélectionnés sur un critère de qualité (auteur(e)s reconnu(e)s, sujet fondamental en ingéniérie, physique ou mathématiques) mais également sur un critère d'utilité.  

  \end{itemize}

\end{frame}
  
\thanksframe{Merci de votre attention !}





\end{document}